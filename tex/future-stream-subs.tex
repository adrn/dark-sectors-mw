% Notes:
% -

% \begin{figure}[!t]
% \begin{center}
% % \includegraphics[width=0.9\textwidth]{visitstats.pdf}
% {\color{red} Figure placeholder}
% \end{center}
% \caption{%
% TODO
% \label{fig:chiplots}
% }
% \end{figure}

\PassOptionsToPackage{usenames,dvipsnames}{xcolor}
\documentclass[modern]{aastex631}
% \documentclass[twocolumn]{aastex631}

% Load common packages
\usepackage{microtype}  % ALWAYS!
\usepackage{amsmath}
\usepackage{amsfonts}
\usepackage{amssymb}
\usepackage{booktabs}
\usepackage{graphicx}
% \usepackage{color}

\usepackage{enumitem}
\setlist[description]{style=unboxed}

\graphicspath{{figures/}}
% Some style hacks:
\renewcommand{\twocolumngrid}{\onecolumngrid}
\setlength{\parindent}{1.1\baselineskip}
\addtolength{\topmargin}{-0.2in}
\addtolength{\textheight}{0.4in}
\sloppy\sloppypar\raggedbottom\frenchspacing

\newcommand{\documentname}{\textsl{Article}}
\newcommand{\sectionname}{Section}
\renewcommand{\figurename}{Figure}
\newcommand{\equationname}{Equation}
\renewcommand{\tablename}{Table}

% Missions
\newcommand{\project}[1]{\textsl{#1}}
\newcommand{\gaia}{\textsl{Gaia}}
\newcommand{\dr}[1]{\acronym{DR}#1}
\newcommand{\apogee}{\acronym{APOGEE}}
\newcommand{\sdss}{\acronym{SDSS}}
\newcommand{\sdssiv}{\acronym{SDSS-IV}}

% Packages / projects / programming
\newcommand{\package}[1]{\textsl{#1}}
\newcommand{\acronym}[1]{{\small{#1}}}
\newcommand{\github}{\package{GitHub}}
\newcommand{\python}{\package{Python}}
\newcommand{\gala}{\project{gala}}
\newcommand{\thejoker}{\project{The~Joker}}

% Stats / probability
\newcommand{\given}{\,|\,}
\newcommand{\norm}{\mathcal{N}}
\newcommand{\pdf}{\textsl{pdf}}

% Maths
\newcommand{\dd}{\mathrm{d}}
\newcommand{\deriv}[2]{\frac{\mathrm{d}{#1}}{\mathrm{d}{#2}}}
\newcommand{\pderiv}[2]{\frac{\partial {#1}}{\partial {#2}}}
\newcommand{\transpose}[1]{{#1}^{\mathsf{T}}}
\newcommand{\inverse}[1]{{#1}^{-1}}
\newcommand{\argmin}{\operatornamewithlimits{argmin}}
\newcommand{\mean}[1]{\left< #1 \right>}

% Non-scalar variables
\renewcommand{\vec}[1]{\ensuremath{\bs{#1}}}
\newcommand{\mat}[1]{\ensuremath{\mathbf{#1}}}

% Unit shortcuts
\newcommand{\msun}{\ensuremath{\mathrm{M}_\odot}}
\newcommand{\mjup}{\ensuremath{\mathrm{M}_{\mathrm{J}}}}
\newcommand{\kms}{\ensuremath{\mathrm{km}~\mathrm{s}^{-1}}}
\newcommand{\mas}{\ensuremath{\mathrm{mas}}}
\newcommand{\muas}{\ensuremath{\mu\mathrm{as}}}
\newcommand{\masyr}{\ensuremath{\mathrm{mas}~\mathrm{yr}^{-1}}}
\newcommand{\mps}{\ensuremath{\mathrm{m}~\mathrm{s}^{-1}}}
\newcommand{\pc}{\ensuremath{\mathrm{pc}}}
\newcommand{\kpc}{\ensuremath{\mathrm{kpc}}}
\newcommand{\kmskpc}{\ensuremath{\mathrm{km}~\mathrm{s}^{-1}~\mathrm{kpc}^{-1}}}
\newcommand{\dayd}{\ensuremath{\mathrm{d}}}
\newcommand{\yr}{\ensuremath{\mathrm{yr}}}
\newcommand{\AU}{\ensuremath{\mathrm{AU}}}
\newcommand{\Kel}{\ensuremath{\mathrm{K}}}

% Misc.
\newcommand{\bs}[1]{\boldsymbol{#1}}

% Astronomy
\newcommand{\DM}{{\rm DM}}
\newcommand{\feh}{\ensuremath{{[{\rm Fe}/{\rm H}]}}}
\newcommand{\mh}{\ensuremath{{[{\rm M}/{\rm H}]}}}
\newcommand{\logg}{\ensuremath{\log g}}
\newcommand{\Teff}{\ensuremath{T_{\textrm{eff}}}}
\newcommand{\vsini}{\ensuremath{v\,\sin i}}
\newcommand{\mtwomin}{\ensuremath{M_{2, {\rm min}}}}
\newcommand{\nbody}{$N$-body}

% Dynamics
\newcommand{\df}{\acronym{DF}}

% TO DO
\newcommand{\todo}[1]{{\color{red} TODO: #1}}
\newcommand{\placeholder}[1]{{\color{purple} #1}}




\shorttitle{}
\shortauthors{Price-Whelan \& Bonaca}

\begin{document}

\title{
    Detecting Dark Matter Subhalos in the Milky Way with Stellar Streams
}

\newcommand{\affcca}{
    Center for Computational Astrophysics, Flatiron Institute, \\
    162 Fifth Ave, New York, NY 10010, USA
}

\newcommand{\affcarnegie}{
    TODO
}

\author[0000-0003-0872-7098]{Adrian~M.~Price-Whelan}
\affiliation{\affcca}
\email{aprice-whelan@flatironinstitute.org}
\correspondingauthor{Adrian M. Price-Whelan}

\author{Ana~Bonaca}
\affiliation{\affcarnegie}


\begin{abstract}\noindent
% Context

% Aims

% Methods

% Results

% Conclusions


\end{abstract}

% \keywords{}

\section{Introduction} \label{sec:intro}

- Dark matter importance / intro.
- Until DM directly detected, and even after, future of constraining DM models is astrophysical
- Know DM halos to $10^8$-–$10^9$ from dwarf galaxies / satellites
- DM models diverge at lower masses - finding low mass substructures would be a huge step forward in constraining DM models.
- Goal of constraining number density of $10^6$ Msun subhalos within reach

- Milky Way is the best place to search for signatures of dark matter substructure.
- Other methods are cool, but don't go as low in halo mass.
- Stellar streams blah blah
- Extragalactic streams far future. Need density contrast. Milky Way is the place to focus effort now.

In this Article, we explore ...

\section{Methods} \label{sec:methods}

We study the observability of dark matter subhalo impacts in stellar streams by running
dynamical simulations of stream--subhalo encounters using a standard ``mock stream''
simulation method coupled with an \nbody\ perturber (the subhalo).
We then simulate observations of stars in these streams by generating a stellar population for the stream with

a and observational uncertainties.


Summary of pipeline: run simulation, use IMF and isochrone to put stellar population on stream model, pick error model and "observe."

\subsection{Stellar Stream + Subhalo Simulations} \label{sec:streams}

We simulate the impact of dark matter subhalos on stellar streams using an approximate
\nbody\ scheme that is implemented using the \gala\ Python package \citep{gala}.

``mock stream'' simulations that approximate the tidal disruption process of a globular cluster

Gala simulations. Setup, etc.

\begin{figure*}[!th]
\begin{center}
\includegraphics[width=\textwidth]{example-image-a} % {illustrate-zvz.pdf}
\end{center}
\caption{%
A gallery of stellar stream models that have been perturbed by dark matter subhalos of
varying mass, all with the same encounter geometry, relative velocity, and fractional
impact parameter (i.e. the impact parameter is a constant factor times the scale radius
of each subhalo, which is set by its mass) shown in sky coordinates oriented with the
stream (longitude $\phi_1$ and latitude $\phi_2$).
The unperturbed stream model is shown in the top panel, and all simulated streams have
the same number of particles.
The progenitor systems are not simulated and the region where the progenitor would be
corresponds to the under-density of star particles near longitude $\phi_1 \sim
-20^\circ$) in each panel.
In all cases (apart from the unperturbed model), the impact site has been rotated to be
at $\phi_1 \approx 0^\circ$.
TODO: takeaway point...
\label{fig:simgallery}
}
\end{figure*}

\subsection{Stellar Populations} \label{sec:stellarpops}

TODO

\section{Results} \label{sec:results}


\section{Discussion} \label{sec:discussion}


\section{Summary and Conclusions} \label{sec:conclusions}


\begin{acknowledgements}

It is a pleasure to thank the CCA Galactic Dynamics Group...


\end{acknowledgements}

\software{
    Astropy \citep{astropy:2013, astropy:2018, astropy:2022},
    gala \citep{gala},
    IPython \citep{ipython},
    numpy \citep{numpy},
    schwimmbad \citep{schwimmbad:2017},
    scipy \citep{scipy}.
}

\bibliographystyle{aasjournal}
\bibliography{fss}

\end{document}
