% Notes:
% -

% \begin{figure}[!t]
% \begin{center}
% % \includegraphics[width=0.9\textwidth]{visitstats.pdf}
% {\color{red} Figure placeholder}
% \end{center}
% \caption{%
% TODO
% \label{fig:chiplots}
% }
% \end{figure}

\PassOptionsToPackage{usenames,dvipsnames}{xcolor}
\documentclass[modern]{aastex631}
% \documentclass[twocolumn]{aastex631}

% Load common packages
\usepackage{microtype}  % ALWAYS!
\usepackage{amsmath}
\usepackage{amsfonts}
\usepackage{amssymb}
\usepackage{booktabs}
\usepackage{graphicx}
% \usepackage{color}

\usepackage{enumitem}
\setlist[description]{style=unboxed}

\graphicspath{{figures/}}
\input{preamble.tex}

\shorttitle{}
\shortauthors{Price-Whelan \& Bonaca}

\begin{document}

\title{
    Detecting Dark Matter Subhalos in the Milky Way with Stellar Streams
}

\newcommand{\affcca}{
    Center for Computational Astrophysics, Flatiron Institute, \\
    162 Fifth Ave, New York, NY 10010, USA
}

\newcommand{\affcarnegie}{
    TODO
}

\author[0000-0003-0872-7098]{Adrian~M.~Price-Whelan}
\affiliation{\affcca}
\email{aprice-whelan@flatironinstitute.org}
\correspondingauthor{Adrian M. Price-Whelan}

\author{Ana~Bonaca}
\affiliation{\affcarnegie}


\begin{abstract}\noindent
% Context

% Aims

% Methods

% Results

% Conclusions


\end{abstract}

% \keywords{}

\section{Introduction} \label{sec:intro}

- Dark matter importance / intro.
- Until DM directly detected, and even after, future of constraining DM models is astrophysical
- Know DM halos to $10^8$-–$10^9$ from dwarf galaxies / satellites
- DM models diverge at lower masses - finding low mass substructures would be a huge step forward in constraining DM models.
- Goal of constraining number density of $10^6$ Msun subhalos within reach

- Milky Way is the best place to search for signatures of dark matter substructure.
- Other methods are cool, but don't go as low in halo mass.
- Stellar streams blah blah
- Extragalactic streams far future. Need density contrast. Milky Way is the place to focus effort now.

In this Article, we explore ...

\section{Methods} \label{sec:methods}

We study the observability of dark matter subhalo impacts in stellar streams by running
dynamical simulations of stream--subhalo encounters using a standard ``mock stream''
simulation method coupled with an \nbody\ perturber (the subhalo).
We then simulate observations of stars in these streams by generating stellar
populations for the streams and applying simple error models to the simulated
observations to generate samples of stars with different faint limits and observational
uncertainties.
We describe each of these steps in more detail in the subsections that follow.
% Summary of pipeline: run simulation, use IMF and isochrone to put stellar population on stream model, pick error model and "observe."

\subsection{Stellar Stream + Subhalo Simulations} \label{sec:streams}

We simulate the impact of dark matter subhalos on stellar streams using an approximate
\nbody\ scheme that is implemented using the \gala\ Python package \citep{gala}.
The streams are simulated using a ``mock stream'' (also known as ``particle spray'')
approach that only simulates the evolution of stars once they have been stripped from a
progenitor globular cluster (i.e. we do not simulate the internal dynamics or disruption
process of the cluster).
Our adopted stream simulation scheme most closely follows the method outlined in
\citet{Fardal:2015}: Given the final phase-space position of a progenitor cluster, we
integrate its orbit backwards in time to some initial time $t_0$.
We then integrate the orbit of the progenitor forward in time from $t_0$ back to the
final time at constant time intervals, but at each time step, we spawn $K_p$ massless
star particles near the instantaneous Lagrange points of the progenitor system with
small dispersions in position and velocity that scale with the adopted initial mass of
the progenitor (i.e. the disruption-time distribution function \df\ depends on the
orbital phase and progenitor mass).
We use the parameter choices and progenitor mass scalings defined in \citet{Fardal:2015}
to set the disruption-time \df.

The ``mock stream'' simulation method outlined above has been demonstrated to reproduce
fully \nbody\ simulated streams in terms of the observed track of the stream in
position--velocity space \citep[e.g.,][]{Kupper, Fardal, Gibbons}.
However, while this approach does capture orbital dynamical effects that impact the
density structure of streams (such as epicyclic over-densities; \citealt{Kupper}), it
does not by default capture more complex density variations induced by bursty disruption
of the progenitor or the internal dynamics of the progenitor.
We note that this method can be adapted to include the self-gravity of the progenitor,
to evolve the disruption-time \df\ as the progenitor mass decreases, and to re-sample
the mass-loss history given a more realistic treatment of the mass-loss rate, all of
which can be used to make the simulated streams better match \nbody\ simulations of
globular cluster streams.
For our purposes we do not do this as we are primarily interested in the observability
of subhalo impacts on streams and not the details of the stream density structure.
The fact that stream internal density structure will confound searches for subhalo
impacts is an important caveat and topic that we discuss in \sectionname~\ref{todo}
below.

For all simulations, we adopt a smooth, four-component Galactic mass model that consists
of a spherical Hernquist nucleus and bulge \citep{Hernquist:1992}, an approximation of
an exponential disk in the radial direction and a $\textrm{sech}^2$ profile in the
vertical direction \citep{Smith:2015}, and a spherical NFW halo \citep{Navarro:1996}.
For the disk model, we adopt a radial scale length $h_R = 2.6~\kpc$ and scale height
$h_z = 300~\pc$ \citep{Bland-Hawthorn:2016}.
We fit for all other parameters (component masses and scale radii) using the same
compilation of Milky Way enclosed mass measurements as used to define the
\texttt{MilkyWayPotential} in \gala \citep{gala}, modified to include more recent
measurements of the disk rotation curve \citep{Eilers:2019}.

We compute orbits of the stream particles and perturbing subhalo using an adaptive 8th
order Runge--Kutta integration scheme \citep{Hairer:1991}.
To densely resolve the particle distribution along each stream, we use a small time step
of $\delta t = 0.25~\Myr$ to simulate the streams (since stream particles are spawned at
each time step), and we release {\color{red} XX} particles sampled from the
disruption-time \df\ at each time step.

For a given choice of progenitor final conditions (i.e. a specification of the position
and velocity of the progenitor at the final time step), we first run a simulation of an
unperturbed stream to serve as a baseline model for comparison.
Figure~\ref{fig:simgallery} (top left) shows an example of an unperturbed stream
simulation, shown in a projection of sky coordinates ($\phi_1, \phi_2$) oriented with
the stream track such that the stream appears approximately at latitude $\phi_2=0^\circ$
relative to the ``sun'' (at $\bs{x} = (-8, 0, 0)~\kpc$ relative to the center of the
mass model).
We use the unperturbed stream simulation to pick a fiducial impact site for a subhalo
encounter, which we then use when re-simulating the stream with a perturber included
(specific experiments and subhalo parameter choices are described below in
Section~{\color{red} TODO}).
We define the impact site as the mean position and velocity of stars in an unperturbed
stream model in a region of the stream that is sufficiently high density by the time
the impact will occur and that is sufficiently far from the progenitor to be less
affected by epicyclic density variations.

\begin{figure*}[!th]
\begin{center}
\includegraphics[width=0.5 \textwidth]{example-image-a} % {illustrate-zvz.pdf}
\end{center}
\caption{%
TODO: Schematic
\label{fig:schematic}
}
\end{figure*}

Once an impact site is chosen, we determine the subhalo orbit by ...

simulate the impact of a subhalo on the stream by

adding the massive subhalo  defining the orbit of the subhalo relative to the impact site using

When simulating a perturbed stream, we define the orbit of the subhalo by specifying its

Gala simulation parameter choices. Potential model. Integrator. Time step. Number of particles. How we stop stream generation and switch to Nbody orbit integration of the stream particles instead.

\begin{figure*}[!th]
\begin{center}
\includegraphics[width=\textwidth]{example-image-a} % {illustrate-zvz.pdf}
\end{center}
\caption{%
A gallery of stellar stream models that have been perturbed by dark matter subhalos of
varying mass, all with the same encounter geometry, relative velocity, and fractional
impact parameter (i.e. the impact parameter is a constant factor times the scale radius
of each subhalo, which is set by its mass) shown in sky coordinates oriented with the
stream (longitude $\phi_1$ and latitude $\phi_2$).
The unperturbed stream model is shown in the top panel, and all simulated streams have
the same number of particles.
The progenitor systems are not simulated and the region where the progenitor would be
corresponds to the under-density of star particles near longitude $\phi_1 \sim
-20^\circ$) in each panel.
In all cases (apart from the unperturbed model), the impact site has been rotated to be
at $\phi_1 \approx 0^\circ$.
TODO: takeaway point...
\label{fig:simgallery}
}
\end{figure*}

\subsection{Stellar Populations} \label{sec:stellarpops}

TODO

\section{Results} \label{sec:results}


\section{Discussion} \label{sec:discussion}


\section{Summary and Conclusions} \label{sec:conclusions}


\begin{acknowledgements}

It is a pleasure to thank the CCA Galactic Dynamics Group...


\end{acknowledgements}

\software{
    Astropy \citep{astropy:2013, astropy:2018, astropy:2022},
    gala \citep{gala},
    IPython \citep{ipython},
    numpy \citep{numpy},
    schwimmbad \citep{schwimmbad:2017},
    scipy \citep{scipy}.
}

\bibliographystyle{aasjournal}
\bibliography{fss}

\end{document}
